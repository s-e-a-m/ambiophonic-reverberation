%!TEX TS-program = xelatex
%!TEX encoding = UTF-8 Unicode
%!TEX root = 2020-GS-ARTICLE.tex
%----------------------------------------------------------------- LANGUAGES ---
\newcommand{\mylanguages}{italian,english} % in reverse order
%---------------------------------------------------------- TITLE & SUBTITLE ---
\newcommand{\mytitle}{Ambiophonic Reverberation}
\newcommand{\mysubtitle}{For a more accessible and less technical introduction
                         to this topic, \\ see Introduction to General Relativity}
%----------------------------------------------------------------- AUTHOR(s) ---
\newcommand{\authorone}{Paradisi Francesco}
\newcommand{\institutione}{Conservatorio S. Cecilia di Roma}
\newcommand{\emailone}{francesco.paradisi10@ gmail.com}
%-------------------------------------------------------------------------------
\newcommand{\authortwo}{Giuseppe Silvi}
\newcommand{\institutiontwo}{Conservatorio Nicolini di Bari}
\newcommand{\emailtwo}{silvi.giuseppe @ docenticonsba.it} % duplicate these 3 lines if more
%-------------------------------------------------------------------------------
\newcommand{\authorthree}{Edoardo Staffa}
\newcommand{\institutionthree}{Conservatorio S. Cecilia di Roma}
\newcommand{\emailthree}{edoardo.staffa1 @ gmail.com} % duplicate these 3 lines if more
%-------------------------------------------------------------- STYLE GS2020 ---
\input{gs2020.tex}
%------------------------------------------------------------ BEGIN DOCUMENT ---
\begin{document}
\maketitle
\thispagestyle{empty}
%-------------------------------------------------------------------- ABSTRACT -
% The abstract is an external txt file inside the includes folder
%-------------------------------------------------------------------------------

\section*{TO DO LIST}
\begin{compactitem}
\item analisi della bibliografia
\item descrizione generica dell'algoritmo di schroeder
\item le potenzialità dell'algoritmo
\item implementazione algoritmo in faust
\item allestimento dell'ambiophonic reverberation in sala concerto
\item implementazione esplicativa per Live Electronics
\item conclusioni
\item
\item
\end{compactitem}

\begin{quote}
  This procedure, which is already obsolete, makes use of delaying devices reproducing not only the discrete initial reflections; but also the reverberation tail. The reflection sequences have herewith to be chosen in such a way that no comb-filter effects, such as flutter echoes, will be produced with impulsive music motifs. The functioning of a simple ambiophonic system can be described as follows: to the direct sound emanating directly from the original source and directly irradiated into the room, there are admixed delayed signals produced by an adequate sound delaying system (in the initial stages this was just a magnetic sound recording system) which are then irradiated like reflections arriving with corre- sponding delay from the walls or the ceiling. This requires additional loudspeakers appropriately distrib- uted in the room for irradiating the delayed sound as diffusely as possible. For further delaying the sound it is possible to arrange an additional feedback from the last output of the delay chain to the input. A system of this kind was first suggested by Kleis51 and was installed realized in several large halls.52,53
  % [51]. D. Kleis, “Moderne Beschallungstechnik” (Modern Sound Reinforcement). Philips tech. Rdsch. 20 (1958/59) 9, S. 272ff. und 21 (1959/60) 3, S. 78ff.
  %
  % [52]. E. Meyer, and H. Kuttruff, “Zur Raumakustik einer großen Festhalle” (Regarding the Room Acoustic of a Large Festival Hall), Acustica 14 (1964) 3, S. 138–147.
  %
  % [53]. W. Kaczorowski, “Urzadzenia elektroakustyczne i naglosnia w Wojewodzkiej Hali Widowiskowo-Sportowej w Katowicach” (Electroacoustic Equipment in the Wojewod Sport Hall in Katowice). Technika Radia i Telewizji 4 (1973) S. 1–16.
\end{quote}

%-------------------------------------------------------------------------------
\subsection*{UNNUMBERED SUB-SECTION}

\begin{warn}[Einstein's theory]
has important astrophysical implications. For example, it
implies the existence of black holes regions of space in which space and time
are distorted in such a way that nothing, not even light, can escape as an
end state for massive stars.
\end{warn}

There is ample evidence that the intense radiation
emitted by certain kinds of astronomical objects is due to black holes. For
example, microquasars and active galactic nuclei result from the presence of
stellar black holes and supermassive black holes, respectively. The bending of
light by gravity can lead to the phenomenon of gravitational lensing, in which
multiple images of the same distant astronomical object are visible in the sky.
General relativity also predicts the existence of gravitational waves, which
have since been observed directly by the physics collaboration LIGO.
%-------------------------------------------------------------------------------
\subsubsection*{UNNUMBERED SUB-SUB-SECTION}
Some predictions of general relativity differ significantly from those of
classical physics, especially concerning the passage of time, the geometry of
space, the motion of bodies in free fall, and the propagation of light. Examples
of such differences include gravitational time dilation, gravitational lensing,
the gravitational redshift of light, and the gravitational time delay. The
predictions of general relativity in relation to classical physics have been
confirmed in all observations and experiments to date. Although general
relativity is not the only relativistic theory of gravity, it is the simplest
theory that is consistent with experimental data. However, unanswered questions
remain, the most fundamental being how general relativity can be reconciled with
the laws of quantum physics to produce a complete and self-consistent theory of
quantum gravity.

Einstein's theory has important astrophysical implications. For example, it
implies the existence of black holes regions of space in which space and time
are distorted in such a way that nothing, not even light, can escape as an
end state for massive stars. There is ample evidence that the intense radiation
emitted by certain kinds of astronomical objects is due to black holes. For
example, microquasars and active galactic nuclei result from the presence of
stellar black holes and supermassive black holes\ldots


\vfill\null

\begin{figure}[b]
\begin{center}
\includegraphics[width=.47\textwidth]{img/image1.png}
\caption{\textbf{Spacetime curvature schematic}. Lattice analogy of the deformation
of spacetime caused by a planetary mass.}
\label{gr01}
\end{center}
\end{figure}

\newpage % USE NEWPAGE TO FORCE COLUMNN INTERRUPTION
%-------------------------------------------------------------------------------
%-------------------------------------------------------------------------------
\section*{UNNUMBERED SECTION}

\begin{quote}
La musica non e` solo composizione. \\
Non è artigianato, non è un mestiere. \\
La musica è pensiero. \cite{nono85}
\end{quote}

Some predictions of general relativity differ significantly from those of
classical physics, especially concerning the passage of time, the geometry of
space, the motion of bodies in free fall, and the propagation of light. Examples
of such differences include gravitational time dilation, gravitational lensing,
the gravitational redshift of light, and the gravitational time delay. The
predictions of general relativity in relation to classical physics have been
confirmed in all observations and experiments to date. Although general
relativity is not the only relativistic theory of gravity, it is the simplest
theory that is consistent with experimental data. However, unanswered questions
remain, the most fundamental being how general relativity can be reconciled with
the laws of quantum physics to produce a complete and self-consistent theory of
quantum gravity.

\begin{table}[htp]
\begin{center}
\begin{tabular}{ll}
\textbf{Stages} & \textbf{Dur.} \\
\hline
\textbf{Omnidirectional Expositions} & 6 mo. \\
Sound-shape analysis and visualizations & \\
Sound-shape reproduction & \\
Sound-shape database design & \\
\hline
\textbf{Micro-Rhythm of sound-shape} & 12 mo. \\
Solo repertoire analysis & \\
Sound-shape explosion in practising & \\
From literature to shapes open-data & \\
\hline
\textbf{Rhythm of sound-shape interactions} & 12 mo. \\
Multiple sources multiple shapes & \\
Relationship and complexity perception & \\
\hline
\textbf{Sound-shape in musical composition} & 12 mo. \\
AI: unleashed writing opportunities & \\
AI: can you listen the time? & \\
\hline
\textbf{Final documentation} & 6 mo. \\
\end{tabular}
\label{timesheet}
\caption{Thinking Tetrahedral Today stages}
\end{center}
\end{table}%

Einstein's theory has important astrophysical implications. For example, it
implies the existence of black holes regions of space in which space and time
are distorted in such a way that nothing, not even light, can escape as an
end state for massive stars. There is ample evidence that the intense radiation
emitted by certain kinds of astronomical objects is due to black holes. For
example, microquasars and active galactic nuclei result from the presence of
stellar black holes and supermassive black holes, respectively. The bending of
light by gravity can lead to the phenomenon of gravitational lensing, in which
multiple images of the same distant astronomical object are visible in the sky.
General relativity also predicts the existence of gravitational waves, which
have since been observed directly by the physics collaboration LIGO. In addition,
general relativity is the basis of current cosmological models of a consistently
expanding universe. \cite{gerzon_70b}

\begin{compactitem}
\item Derivations of the Lorentz transformations
\item Einstein–Hilbert action
\item Tests of general relativity
\item Two-body problem in general relativity
\end{compactitem}

\begin{figure}[t]
\centering
\includegraphics[width=.47\textwidth]{img/image2.jpg}
\caption{Mind Mapping}
\label{gs}
\end{figure}

\begin{equation}
m(x,p,\theta) = (p*x) + ((1-p)*(x\cos\theta)
\label{eq:mid}
\end{equation}

Some predictions of general relativity differ significantly from those of
classical physics, especially concerning the passage of time, the geometry of
space, the motion of bodies in free fall, and the propagation of light.

%--------------------------------------------
%----------------larghezza massima del codice
\begin{lstlisting}
mspan(x,p,rad) = m,s
with{
  m = (p*x)+((1-p)*(x*cos(rad)));
  s = x*(sin(-rad));
};
\end{lstlisting}

Examples of such differences include gravitational time dilation, gravitational
lensing, the gravitational redshift of light, and the gravitational time delay.
The predictions of general relativity in relation to classical physics have been
confirmed in all observations and experiments to date.

\vfill\null

\raggedright
\bibliographystyle{unsrt}
\bibliography{includes/bibliography.bib}

\end{document}

%%%%%%%%%%%%%%%%%%%%%%%%%%%%%%%%%%%%%%%%%%%%%%%%%%%%%%%%%%%%%%%%%%%%%%%%%%%%%%%%
% 2020 GIUSEPPE SILVI ARTICLE TEMPLATE BASED ON
%%%%%%%%%%%%%%%%%%%%%%%%%%%%%%%%%%%%%%%%%%%%%%%%%%%%%%%%%%%%%%%%%%%%%%%%%%%%%%%%
% Journal Article
% LaTeX Template
% Version 1.4 (15/5/16)
% This template has been downloaded from:
% http://www.LaTeXTemplates.com
% Original author:
% Frits Wenneker (http://www.howtotex.com) with extensive modifications by
% Vel (vel@LaTeXTemplates.com)
% License:
% CC BY-NC-SA 3.0 (http://creativecommons.org/licenses/by-nc-sa/3.0/)
%%%%%%%%%%%%%%%%%%%%%%%%%%%%%%%%%%%%%%%%%%%%%%%%%%%%%%%%%%%%%%%%%%%%%%%%%%%%%%%%
